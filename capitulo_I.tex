\begin{doublespace}

\part{Introducci\'{o}n.}
\end{doublespace}
\begin{doublespace}

\section{Planteamiento del problema.}
\end{doublespace}

En la actualidad la electr\'{o}nica est\'{a} presente en pr\'{a}cticamente
en todos los aspectos de nuestra vida a trav\'{e}s de una gran infinidad
de dispositivos y sistemas: tel\'{e}fonos inteligentes, monitores
de ritmo cardiaco, c\'{a}maras fotogr\'{a}ficas, televisores, autom\'{o}viles,
refrigeradores, computadoras, etc. 

Todos estos dispositivos realizan de manera interna la manipulaci\'{o}n
e interpretaci\'{o}n de se\~{n}ales el\'{e}ctricas, que en otras palabras
es lo que se conoce como procesamiento de se\~{n}ales. El procesamiento
de una se\~{n}al puede aplicarse, por ejemplo, en el reconocimiento
de voz para determinar qui\'{e}n es la persona que habla; para determinar,
mediante una imagen, piezas defectuosas en una l\'{\i}nea de producci\'{o}n
o para la protecci\'{o}n de informaci\'{o}n (encriptaci\'{o}n). 

El procesamiento de se\~{n}ales involucra la realizaci\'{o}n de operaciones
matem\'{a}ticas sobre las se\~{n}ales, las cuales son llevadas por
sistemas cuya \'{u}nica funci\'{o}n es precisamente el llevar a cabo
esas operaciones, los procesadores digitales de se\~{n}ales (DSP,
por sus siglas en ingl\'{e}s) y los arreglos programables (FPGA, por
sus siglas en ingl\'{e}s) son los encargados de ello. 

En muchas aplicaciones de procesamiento de se\~{n}ales se requiere
una velocidad de procesamiento elevada (por ejemplo procesamiento
de video) por lo que, debido al paralelismo de su operaci\'{o}n, los
FPGA son aptos para ser utilizados en ellas\cite{kehtarnavaz_digital_2010}. 

Con el fin de explotar las ventajas que los FPGA poseen y poderlos
aplicar de una manera eficaz en el procesamiento de se\~{n}ales es
necesario contar con s\'{o}lidos conocimientos principalmente en metodolog\'{\i}as
de dise\~{n}o digital e implementaci\'{o}n matem\'{a}tica de algoritmos;
estos conocimientos deben adquirirse desde la academia, puesto que
es el tiempo ideal en que el futuro ingeniero o arquitecto de sistemas
puede ir desarrollando, a trav\'{e}s de la experimentaci\'{o}n, las
habilidades necesarias para crear prototipos en el que se involucre
el procesamiento de se\~{n}ales. 

Teniendo como objetivo principal el recortar la curva de aprendizaje,
las empresas l\'{\i}deres en FPGA como Xilinx, Altera y Synopsys proporcionan
plataformas de trabajo que puede interactuar con Matlab (software
especializado que permite la implementaci\'{o}n y prueba de algoritmos).
De esta forma, el alumno puede poner en pr\'{a}ctica de forma \'{a}gil
y sin complicaciones los conocimientos adquiridos en las \'{a}reas
de procesamiento digital de se\~{n}ales. 

Muchas veces la informaci\'{o}n que el fabricante proporciona sobre
sus plataformas de trabajo es escasa y poco concreta, lo que puede
impactar negativamente en el inter\'{e}s del alumno, provocando que
los conocimientos y conceptos no queden del todo entendidos. 

\section{Revisi\'{o}n de la literatura.}

\section{Prop\'{o}sito.}
\begin{doublespace}

\subsection{Objetivo general.}
\end{doublespace}
\begin{itemize}
\begin{doublespace}
\item Describir el proceso de implementaci\'{o}n de un sistema de procesamiento
de se\~{n}ales e im\'{a}genes mediante hardware reconfigurable (FPGA)
y la tarjeta de desarrollo Atlys.
\end{doublespace}
\end{itemize}
\begin{doublespace}

\subsection{Objetivos espec\'{\i}ficos.}
\end{doublespace}
\begin{itemize}
\begin{doublespace}
\item Facilitar el dise\~{n}o e implementaci\'{o}n de un sistema de procesamiento
de audio en tiempo real, basado en el desarrollo de un algoritmo de
eco, as\'{\i} como un sistema de detecci\'{o}n de bordes en una imagen
basado en el algoritmo Sobel, ambos utilizando bloques de Xilinx System
Generator para Simulink.
\item Mostrar la conversi\'{o}n de los algoritmos matem\'{a}ticos b\'{a}sicos
que intervienen en el procesamiento de se\~{n}ales, a hardware en
FPGA, haciendo uso de la abstracci\'{o}n que proporciona Simulink
\item Describir las t\'{e}cnicas de implementaci\'{o}n m\'{a}s eficientes
para obtener el mayor rendimiento sobre la familia FPGA Spartan 6
utilizada en la tarjeta Atlys
\item Dise\~{n}ar las propiedades intelectuales (IPs) m\'{a}s comunes en
el tratamiento de se\~{n}ales tales como bloques de filtros FIR, IIR
y convoluciones, utilizando los entornos de programaci\'{o}n de MATLAB\textregistered{}
y Xilinx\textregistered .
\item Explicar los diferentes m\'{e}todos de ejecuci\'{o}n del hardware
dise\~{n}ado en Simulink\textregistered , sobre la tarjeta Atlys. 
\item Justificar el uso de MATLAB/Simulink\textregistered{} y Xilinx/ISE\textregistered{}
para el dise\~{n}o e implementaci\'{o}n de algoritmos complejos en
contraste con el uso tradicional de HDL puro.
\end{doublespace}
\end{itemize}
\newpage{}
